\chapter*{Conclusion}
\addcontentsline{toc}{chapter}{Conclusion}
\markboth{Conclusion}{}

Modelling is a field of soaring interest at the interface between engineering and medicine. Accurate modelling has the potential to enhance diagnostic accuracy and efficiency, to provide a better understanding of physiological behaviours in the body with reduced costs and to avoid traditional invasive methods. In breathing, there is a crucial need for a non-invasive means to assess pulmonary function among infants, the elderly, or patients for which conventional procedures would impose a compromise in health; for example, patients in intensive care units. 

In this thesis, we presented a novel non-invasive method for performing pulmonary function testing using structured light (SLP), 3D modelling and optimisation algorithms. In contrast to related breathing simulation models, our simulation is geared towards medical applications. As a consequence, our model offers significant improvements and a higher level of sophistication compared to others in the published literature: it has high anatomical accuracy, is fully tunable and can fit different patient anatomies according to various given characteristics. In addition, the optimisation process used to provide data-driven simulations has not previously been well explored. Only DiLorenzo \cite{dilorenzo2009breathing} proposed a sketch of a data-driven approach in his \emph{Breathe Easy} model. In addition, we believe it to be very important that our method be rigorously tested for equivalence to another well-acknowledged lung volume measuring technique: spirometry. Thus, we did not only perform the validation exercise using the tools common in most medical comparisons studies, we also utilised state-of-the-art frequentist and Bayesian tests that use the underlying nature and characteristics of our data, providing testing techniques that are much more appropriate. The results themselves provide strong evidence that the simulation and spirometry data are very similar but also point out that they are not exactly the same. Some proposals for future extensions to the work presented are explored below.

Even though the assumptions made about the nature of the activation inputs (sine waves) of the respiratory muscles provide good results for the simulated changes of lung volume, we believe that they are simplifications of the true nature of the respiratory muscle activations. Ideally, we would operate the optimisation over a basis of functions---which could be easily implemented in our code---but the computational power that would be required is for the moment not available to us.

One advantage of our method is that it offers the possibility to spatially assess breathing patterns. We have already exploited this feature through the natural segmentation of our model: the rib cage and abdomen. This enabled us to analyse the contributions of the different activation magnitudes and phase shifts that operate in SLP datasets. However, we could improve the compartmental analysis further by splitting up the chest wall in many other ways. For example, splitting the chest wall into left-hand and right-hand sides could prove interesting for patients who have suffered from a collapsed lung; or we could isolate specific rib regions to reproduce the breathing style of a patient who had a tumour resection in his chest wall or even a rib replacement. This could be done by selecting the different muscles involved and modifying their different intrinsic parameters and activation inputs.

Additional validation tests need to be performed. In particular, it is desirable that we perform a proper analysis of our method on different ranges of the population: the young, the elderly, women etc. For the moment, we have performed our validation on mostly healthy male subjects but we have all the technical tools in hand to extend our database.

One of the major advantages that the SLP technique brings is that it does not require physical contact with a patient. As such, the system could be used to monitor premature babies inside incubators or animals. To do so, we would need to adapt the rigid skeleton and several muscle layers to fit baby and animal anatomies in order to apply our technique; but the general principle and main steps would be exactly the same.

Another area of future work lies at the implementation of the technique itself. Currently, the optimisation process takes around 10 hours to have a data-driven simulation of  20 seconds. A port of the current Matlab and MEL implementation to compiled C/C++ code and above all, a faster but equally accurate rigid solver as the one currently used (Runge Kutta Adaptive), are needed for this.